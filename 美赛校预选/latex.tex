\documentclass[12pt]{article}
\usepackage{geometry}
\geometry{left=2cm,right=2cm,top=2.8cm,bottom=2cm}
\setlength{\headheight}{18pt}
\setlength{\headsep}{18pt}
\setlength{\footskip}{25pt}


%\usepackage{mathptmx}
\usepackage{newtxtext}
%\usepackage{fontspec}
%\setmainfont{Times New Roman}

\usepackage{tocloft}
\usepackage{hyperref}
\usepackage{amsmath,amssymb,amsthm}
\usepackage{parskip}
\usepackage{lipsum}
\usepackage{booktabs}
\usepackage{float}

%\usepackage[pdftex]{graphicx}
\usepackage{graphicx}
\usepackage{xcolor}
\usepackage{tabularx}
\usepackage{booktabs}
\usepackage{array}
\usepackage{makecell}
\usepackage{fancyhdr}
\usepackage{url}
\lhead{Team}
\rhead{}

% setting of table of contents
\renewcommand\cftsecaftersnum{.}
\renewcommand\cftsubsecaftersnum{.}
\renewcommand\cftsubsubsecaftersnum{.}

% setup hyperlink
\hypersetup{
  colorlinks=true,
  linkcolor=black,
  citecolor=blue,
  filecolor=blue,
  urlcolor=blue
}

% graphic path
\graphicspath{{./figures/}}
\DeclareGraphicsExtensions{.pdf, .jpg, .tif, .png}

% package for section etc.
\usepackage{titlesec}
\titleformat{\section}{\Large\bfseries}{\thesection.}{1.2ex}{}
\titleformat{\subsection}{\large\bfseries}{\thesubsection.}{1.2ex}{}
\titleformat{\subsubsection}{\normalfont\itshape}{\normalfont\thesubsubsection.}{1.2ex}{}

% command for line spacing
%\renewcommand{\baselinestretch}{1.1}

\setlength{\parskip}{1em}

\newtheorem{theorem}{Theorem}
\newtheorem{corollary}[theorem]{Corollary}
\newtheorem{lemma}[theorem]{Lemma}
\newtheorem{definition}{Definition}

%%%%%%%%%%%%%%%%%%%%%%%%%%%%%%%%
\begin{document}

\thispagestyle{empty}

\medskip
\begin{center}
\large\bfseries Summary
\end{center}
The popularity of electric buses represents the possibility of further progress towards the goal of sustainable development of 
urban transportation. This paper collected the city characteristic data of St. Louis, established carbon emission model, atmosphere 
model, life cycle cost model, cost-benefit model and comprehensive evaluation model, introduced linear programming, analytic hierarchy 
process and other methods, comprehensively analyzed the ecological and economic benefits of electric buses, and helped the city plan 
the update roadmap of electric buses, which has strong practical significance.

For problem 1, we need to analyze the ecological benefits that electric bus fleets can bring. From a practical point of 
view, electric buses use electric energy as a power source, and the operation process has the environmental protection characteristics 
of zero emission of pollutants. We determined that the object of the study is the most important factor affecting the ecological 
environment emitted by diesel vehicles, namely carbon dioxide. Carbon emission model was established, performance data of electric 
bus and diesel bus were collected, linear analysis method was introduced to calculate the total annual carbon emission of each bus, 
and benefit analysis was conducted by comparing the data of the two. Secondly, considering that the damage of diesel vehicles to the 
ecological environment is partly due to the emission of air pollutants, we established an atmospheric model and calculated the total 
emissions of different types of air pollutants of the two types of buses in the service life, and drew a conclusion from the analysis.

For problem 2, what we need to analyze is the financial cost-benefit problem. From the current industry development point of view, 
the operating costs of electric buses are decreasing, which is reflected in the reduction of battery prices and the continuous 
investment of government financial subsidies, which is cost-effective. Therefore, we first established the life cycle cost model, 
divided the cost generated in the life cycle of the two types of buses into two types, and analyzed and compared the cost difference 
between the two. Secondly, a cost-benefit model is established to analyze the benefits of the electric bus fleet and calculate its 
total upfront investment cost. The net present value income analysis method is introduced to calculate the net present value income 
that can be produced during the service life cycle of electric buses, and the economic benefits of electric bus operation are analyzed 
from the data.

For problem 3, we need to create a 10-year roadmap to help St. Louis plan the renewal of its electric bus fleet. 
After analysis, we divide the project cycle into four parts. During the initial assessment and planning period, we conduct market 
research and infrastructure evaluation and identify government financial subsidy options. In the pilot stage, a small number of 
electric buses are put into pilot use on selected routes, and the operation data of electric buses and passenger feedback are 
collected, so as to determine the detailed operation plan. In the phased deployment phase, the main problem is to solve the 
infrastructure problem and optimize the route. In the comprehensive transformation phase, the entire bus fleet will be electrified 
and a long-term vehicle maintenance update mechanism and a regular fleet evaluation mechanism will be established. Therefore, 
we establish a comprehensive evaluation model, introduce linear programming analysis method, determine the weight of factors 
affecting route planning, and combine the actual data to determine the optimal deployment route of electric buses.


%关键词
\vspace{0.4cm}
\noindent \textbf{Keywords: }\LaTeX,~  HiMCM/MidMCM,~ Champion

\newpage
\tableofcontents
\thispagestyle{empty}
\newpage

%正文部分
\pagestyle{fancy}
\rhead{Page \thepage~of~25}
\setcounter{page}{1}
\section{Introduction}
\subsection{Restatement of the Problem}
With the continuous deterioration of the ecological environment, more and more people are aware of the importance of environmental 
protection and practice environmental protection in all aspects of life, and public transportation has become the choice of more 
and more people. At present, most buses still use diesel as a power source, but some cities have put electric buses into operation, 
which has a huge boost to energy conservation, emission reduction and pollution reduction. More and more countries and regions are 
aware of the superiority of electric buses, and are adopting policies to vigorously support the development of electric bus industry.

However, challenges include high initial costs, charging infrastructure development,lengthy charging times, and potential range 
limitations.

Based on this situation, we need to solve the following problems:

\textbullet{Construct a model to aid cities in understanding the ecological consequences of transitioning to an all-electric bus fleet.}

\textbullet{Construct a model that focuses on the financial implications associated with a conversion to e-buses.}

\textbullet{Transportation officials in metropolitan areas are exploring approaches in which they
gradually change their fleet from combustion engines buses to electric. Assuming the goal
is to have a fully electric fleet no later than 2033, utilize your previously developed
models to craft a 10-year roadmap that urban transport authorities can leverage to plan
their e-bus fleet updates.}

\textbullet{Write a one-page letter to the transportation officials of one of your chosen
metropolitan areas in which you detail your recommendation for their transition to
e-buses.}

\subsection{Overview of Our Work} %问题分析

\subsubsection{Problem 1:Ecological Benefits Analysis}
This question mainly studies the ecological benefits brought by electric buses. After analysis, we decided to directly reflect the 
impact of electric buses and diesel vehicles on the ecological environment through detailed data and then compare the ecological 
benefits brought by electric buses. First of all, considering that the main substance emitted by bus operation is carbon dioxide, 
we establish a carbon emission model, analyze the carbon emission size of the two types of vehicles combined with the collected 
performance data, and compare the results. Secondly, considering the impact of global warming, cars will also cause air pollution. 
Therefore, we established an atmospheric model to collect pollutant emission factors of other harmful substances in the two types of 
cars except carbon dioxide, and then calculated the emissions of different kinds of pollutants during the operation of the two types 
of buses, and finally conducted data analysis and comparison.

\subsubsection{Problem 2:Cost-effectiveness Analysis}
The purpose of this question is to find out whether electric buses are competitive in the market compared with diesel buses in terms 
of both costs and benefits, that is, to analyze whether the economic benefits of electric buses are worthy of large-scale use by the 
local government. First of all, considering that both electric buses and diesel buses have service life, and the process of replacing
diesel buses with electric buses is gradual, we established a life cycle cost model, determined the life cycle length as the service 
life of buses, and roughly divided the cost into acquisition cost and maintenance cost within the life cycle. Combined with the 
collected performance parameters of the two types of vehicles and the relevant city characteristics data of St. Louis, the various 
costs generated during the life cycle of the two types of vehicles were simply summed up and compared, and the market competitiveness 
of the two types of buses was compared. Secondly, the cost-benefit analysis model is established, the net present value calculation 
formula is introduced, the income is converted into data, and the economic benefit of electric buses is judged by judging the size of 
the data.

\subsubsection{Problem 3:Optimal Roadmap Planning}
This question requires us to develop a 10-year roadmap for the city's transportation authorities to help them update the city's 
electric bus fleet. First of all, we make it clear that due to the characteristics of electric buses themselves, problems such as 
infrastructure, public acceptance, and available funds need to be solved before they are put into trial use, and the completion of 
the all-electric bus fleet needs to be carried out gradually from the pilot stage where a small number of buses are put into use. 
Therefore, we will divide the 10-year renewal cycle into four stages, and will complete the renewal and mature use of the all-electric 
bus fleet at the end of these four stages. We establish a comprehensive evaluation model, enumerates multiple factors affecting route 
planning, and introduces analytic hierarchy process (AHP) to confirm the respective weights of the influencing factors, so as to 
achieve the goal of maximum passenger flow coverage and minimum operating cost in the roadmap planning.

\begin{figure}[H]
	\centering
	\includegraphics[scale=0.7]{flowchart.pdf}% 图片相对位置
	\caption{Flowchart for this Article} % 图片标题
	\label{1} 
\end{figure}


\section{Assumptions and Justifications}
We make the following assumptions to help us with our modeling. These assumptions form the background for our subsequent analysis.

\textbullet{\bfseries{Assumption 1:the bus will run normally, independent of personal factors such as the driver and other small 
probability events such as extreme weather.}}This is because we need ensure that the many factors in the model such as service 
frequency, emissions, etc., do not produce abnormal extreme values.

\textbullet{\bfseries{Assumption 2:the investment plan of electric buses is not affected by external factors and is continuously 
supported by policies.}}This is done to ensure that the model is built according to the established plan.

\textbullet{\bfseries{Assumption 3:the purchase and maintenance cost of buses changes according to the constant annual growth rate.}}

\textbullet{\bfseries{Assumption 4:the electric bus does not have a small probability event such as a car accident and other abnormal 
circumstances in 10 years, resulting in the change of the given number and revenue cost.}}This is to reduce the impact of small 
probability events on the model.

\section{Notations}
\begin{table}[!htbp]
\centering
    \begin{tabular}{ccc}
	\toprule
	Symbol&Definition&Unit\\
	\midrule
	\(x_i\)&Evaluation indicator&-\\
	\({\tilde x_i}\)&Standardized indicators&-\\
	\({\mu _i}\)&Average value&-\\
	\(s_i\)&Standard deviation&-\\
	\({R}\)&Correlation coefficient matrix&-\\
	\(y_i\)&Principal Components&-\\
	\(b_i\)&The information contribution of eigenvalue&-\\
	\(T\)&Composite score&-\\
	\bottomrule
    \end{tabular}
\end{table}


\section{Problem 1:Ecological Benefits Analysis}
\subsection{Comparison of Carbon Emission Based on Carbon Emission Model}\
\subsubsection{Establishment of Model}
\subsubsection{Data Analysis}

\subsection{Comparison of Pollutant Emissions Based on Atmospheric Model}
\subsubsection{Establishment of Model}
\subsubsection{Data Analysis}

\section{Problem 2:Cost-effectiveness Analysis}
\subsection{Cost Comparison Based on Life Cycle Model}
\subsubsection{Establishment of Model}
\subsubsection{Data Analysis}

\subsection{Economic Benefit Analysis Based on Cost-benefit Analysis Model}
\subsubsection{Establishment of Model}
\subsubsection{Data Analysis}

\section{Problem 3:Optimal Roadmap Planning Problems Based on Comprehensive Evaluation Model}
\subsection{Ten Year Work Deploment Arrangement}
\textbullet{Stage 1:}
\textbullet{Stage 2:}
\textbullet{Stage 3:}
\textbullet{Stage 4:}
\subsection{Optimal Roadmap Planning Based on Comprehensive Evaluation Model}
\subsubsection{Establishment of Model}
\subsubsection{Data Analysis}


\newpage
\section{A Letter to the Transportation Officials}
\lipsum[1-2]

\section{Sensitivity Analysis}
\lipsum[1-2]


\section{Model Evaluation and Further Discussion}

\subsection{Advantages of Model}

\subsection{Disadvantages of Model}


\newpage
\begin{thebibliography}{100}
\bibitem{1}\url{https://afdc.energy.gov/vehicles/electric_emissions.html}
\bibitem{2}\url{https://about.bnef.com/electric-vehicle-outlook/}
\bibitem{3}\url{https://www.apta.com/research-technical-resources/transit-statistics/electric-bus-fact-sheets/}
\end{thebibliography}

\end{document}